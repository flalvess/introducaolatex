\section{Índices}

A criação de índices é bastante simples. Inclua 
no preâmbulo:


\begin{verbatim}
\usepackage{makeidx}
\makeindex
\end{verbatim}

e cite usando:

\begin{verbatim}
\index{nome}
\end{verbatim}

Para mostrar o índice, inclua:

\begin{verbatim}
\printindex
\end{verbatim}

antes do fim do seu documento. O índice é processado por um outro comando, \verb+makeindex+: \verb+makeindex arquivo.tex+.

Também é possível inserir subentradas, separando o nível superior do inferior com uma exclamação. Por exemplo, para inserir uma entrada ``aves'': \verb+\index{Aves}+ e uma subentrada ``canários'': \verb+\index{Aves!Canários}+. 

Também é possível fazer uma referência a outra entrada: \verb+\index{Canários|see {Aves}}+, que aparecerá como: ``Canários, \textit{veja} Aves''; ou \verb+\index{Canários|seealso {Aves}}+, que aparecerá como: ``Canários, \textit{veja também} Aves'';

Para formatar uma entrada do índice, deve-se seguir o formato: \verb+\index{Aves@\textbf{Aves}}+, ou \verb+\index{Aves@\textit{Aves}}.