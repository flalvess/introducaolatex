\section{Utilizando o pacote \texttt{biblatex}}

%Exemplos dos estilos podem ser encontrados em:
%\url{http://www.ctan.org/tex-archive/macros/latex/contrib/biblatex//doc/examples}.

O pacote \texttt{biblatex} fornece um substituto tanto para o \texttt{natbib} quanto para o \texttt{bibtex}. Sua vantagem é a maior facilidade e versatilidade nas citações, e a capacidade de aceitar entradas em codificação \texttt{utf8}. Um arquivo \texttt{biblatex} é basicamente arquivo \texttt{.bib}. Ele pode ser processado com o comando \texttt{bibtex} (utilizando a opção \texttt{backend=bibtex}, que é padrão) ou com o comando \texttt{biber} utilizando \texttt{backend=biber} como opção. O \texttt{biblatex} inclui muitas opções, a abaixo apresentaremos somente algumas. (Para mais detalhes, ver a documentação do pacote.) A desvantagem do \texttt{biblatex} é que não há tantos estilos disponíveis quanto para \texttt{bibtex}. Para estilos disponíveis por padrão, ver: \url{http://www.ctan.org/tex-archive/macros/latex/exptl/biblatex-contrib}.


\begin{verbatim}
\usepackage[style=historian,citestyle=authoryear,natbib=true,
doi=false,backend=bibtex]{biblatex}
\usepackage{csquotes}
\DefineBibliographyStrings{brazil}{namedash={---------}}
\addbibresource{latex}

\begin{document}
(documento)

\printbibliography

\end{document}

\end{verbatim}

\begin{description}
\item[style=historian] O estilo de referência a aparecer na bibliografia.
\item[citestyle=authoryear] O estilo de citação no texto. Os estilos básicos são \texttt{citestyle=authoryear} para autor-data e \texttt{citestyle=numeric} para numérico, mas várias outras opções menos usuais estão disponíveis.
\item[natbib=true] Para compatibilidade com natbib.
\item[doi=false] Para retirar entrada \textsc{DOI} (Digital Object Identifier -- \url{http://en.wikipedia.org/wiki/Digital_object_identifier}).
\item[backend=bibtex] Utiliza o \texttt{bibtex} para compilar.
\item[csquotes] Útil para citações utilizando aspas em várias línguas, altamente recomendável para bom funcionamento do \texttt{biblatex}.
\item[DefineBibliographyStrings\ldots] Por padrão, o nome do autor, quando repetido, é substituído no sistema autor-data por um travessão (---). Aqui temos o nome do idioma e o tamanho dessa linha para autores repetidos.
\item[addbibresource] Aqui colocamos o arquivo \texttt{.bib} (a base de dados bibliográfica) a ser utilizada. (Note a diferença com \texttt{bibtex}, onde essa informação era dada no fim do documento.)
\item[printbibliography] Para gerar a bibliografia (antes do fim do documento).
\item[damedash=false] Não substituir o nome do autor repetido por uma linha/travessão.
\end{description}

\subsection{Citações utilizando \texttt{biblatex}}

\subsubsection*{\texttt{cite}}
Utiliza citações diretas, sem parênteses. Exemplos:


\begin{tabular}{cc}
\centering
\verb+\cite{lamport94}+ &  Lamport, 1994.\\

\verb+\cite[59]{lamport94}+ & Lamport, 1994, p. 59.\\

\verb+\cite[ver][]{lamport94}+] & ver Lamport, 1994.\\

\verb+\cite[ver][59--63]{lamport94}+ & ver Lamport, 1994, p. 59--63.\\
\end{tabular}

\subsubsection*{\texttt{parencite}}
Utiliza citações entre parênteses. Exemplos:


\begin{tabular}{cc}
\centering
\verb+\parencite{lamport94}+ & \parencite{lamport94}.\\

\verb+\parencite[59]{lamport94}+ & \parencite[59]{lamport94}.\\

\verb+\parencite[ver][]{lamport94}+ &  \parencite[ver][]{lamport94}.\\

\verb+\parencite[ver][59--63]{lamport94}+ &  \parencite[ver][59--63]{lamport94}.\\
\end{tabular}

\subsubsection*{\texttt{parencite*}}
Coloca somente o ano da em parênteses. Pode ser utilizado em conjunção com \verb+\citeauthor{}+.

\begin{center}
\verb+\parencite*{lamport94}+\qquad

 \parencite*{lamport94}.
\end{center}

\subsubsection*{\texttt{footcite}}
Coloca as referências como notas de rodapé. Aceita as opções exemplificadas acima. Exemplo:


\begin{center}
\verb+\footcite{lamport94}+\qquad
 Nota de rodapé.\footcite{lamport94}
\end{center}


\subsubsection*{\texttt{textcite}}
Para nome do autor no texto e o resto entre parênteses:

\begin{tabular}{cc}
\centering
\verb+\textcite{pohlen_letter_2011}+ & \textcite{pohlen_letter_2011} mostram\ldots .\\

\verb+\textcite[59]{pohlen_letter_2011}+ & \textcite[59]{pohlen_letter_2011} mostram\ldots .\\

\end{tabular}

\subsubsection*{Citações múltiplas}
Para organizá-las, utilize a opção \texttt{sortcites} do \texttt{biblatex}. Caso contrário, aparecerão na ordem indicada.

\subsubsection*{Abreviações}
Abreviações de obras podem aparecer, por exemplo, antes da bibliografia propriamente dita. Útil para edições críticas e citações reiteradas a obras inteiras.

Para lista de obras abreviadas, insira: \verb+\printshorthands+. As entradas no arquivo de bibliografia devem conter \verb+shorthand={}+, que indicará como a abreviação aparecerá no texto. Por exemplo:\footnote{Exemplos retirados do arquivo: \url{http://linorg.usp.br/CTAN/macros/latex/contrib/biblatex/doc/examples/50-style-authoryear.tex}.}

\begin{verbatim}
@inbook{kant:kpv,
  shorthand = {KpV},
  hyphenation = {german},
  author = {Kant, Immanuel},
  bookauthor = {Kant, Immanuel},
  title = {Kritik der praktischen Vernunft},
  shorttitle = {Kritik der praktischen Vernunft},
  booktitle = {Kritik der praktischen Vernunft. Kritik der Urtheilskraft},
  maintitle = {Kants Werke. Akademie Textausgabe},
  volume = {5},
  publisher = {Walter de Gruyter},
  location = {Berlin},
  date = {1968},
  pages = {1--163},
  }

@inbook{kant:ku,
  shorthand = {KU},
  hyphenation = {german},
  author = {Kant, Immanuel},
  bookauthor = {Kant, Immanuel},
  title = {Kritik der Urtheilskraft},
  booktitle = {Kritik der praktischen Vernunft. Kritik der Urtheilskraft},
  maintitle = {Kants Werke. Akademie Textausgabe},
  volume = {5},
  publisher = {Walter de Gruyter},
  location = {Berlin},
  date = {1968},
  pages = {165--485},
  }
\end{verbatim}

Citando:


\begin{center}
\verb+Obras de Kant: \cite{kant:kpv,kant:ku}.+
\qquad
Obras de Kant: \cite{kant:kpv,kant:ku}.
\end{center}

