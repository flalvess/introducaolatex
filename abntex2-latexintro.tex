\chapter{Documentos técnicos e científicos brasileiros com abnTeX2}


%\begin{center}
%\begin{mdframed}
%Advertência: os modelos \textsc{abnt} não geram nada muito aceitável tipograficamente. Se não precisar seguir essa regra, não siga.
%\end{mdframed}
%\end{center}

O abn\TeX 2, evolução do abn\TeX{} (''Absurd Norms for TeX"), é uma suíte composta por uma classe e por pacotes de citação e de formatação de estilos bibliográficos para \LaTeX{} que atendem os requisitos das normas da \textsc{abnt} (Associação Brasileira de Normas Técnicas) para elaboração de documentos técnicos e científicos brasileiros, como artigos científicos, relatórios técnicos, trabalhos acadêmicos como teses, dissertações, projetos de pesquisa e outros documentos do gênero. (Este documento não foi escrito sob o modelo do abn\TeX, nem segue os
padrões da \textsc{abnt}.)

O projeto é hospedado em: \url{http://code.google.com/p/abntex2/} e também está disponível em várias distribuições (\TeX Live, Mik\TeX, Mac\TeX{})

A instalação da classe abn\TeX 2 é simples, semelhante a outros pacotes ou classes (ver seção \ref{pacotes}). Instruções detalhadas para vários sistemas e distribuições podem ser encontradas em: \url{https://code.google.com/p/abntex2/wiki/Instalacao}.

A suíte abn\TeX 2 é composta por três elementos principais:\footnote{Somam-se a
esses elementos a documentação e os modelos (disponíveis em: \url{https://code.google.com/p/abntex2/downloads/list}), todos distribuídos de forma
conjunta.}

\begin{itemize}
  \item a classe de formação de trabalhos acadêmicos \textsf{abntex2};
  \item o pacote de citações bibliográficas \texttt{abntex2cite}; e
  \item as especificações de formatação de referências bibliográficas
  \texttt{abntex2-num.bst} e \texttt{abntex2-alf.sty}.
\end{itemize}

%O abnTeX2 foi desenvolvido com base nos requisitos das seguintes normas ABNT:
%
%\begin{description}
%  \item[ABNT NBR 6022:2003] Informação e documentação - Artigo em publicação
%  periódica científica impressa - Apresentação
%  \item[ABNT NBR 6023:2000] Informação e documentação - Referência -
%  Elaboração\footnote{A versão corrente da ABNT NBR 6023 é a 6023:2002. Porém,
%  até este momento, o abnTeX2 traz os estilos compatíveis com a versão
%  anterior da norma. A atualização dos estilos é uma das etapas posteriores do
%  projeto. Consulte a documentação do pacote \texttt{abntex2cite} \texttt{abntex2cite-alf}
%  para mais informações.}
%  \item[ABNT NBR 6024:2012] Informação e documentação - Numeração
%  progressiva das seções de um documento - Apresentação
%  \item[ABNT NBR 6027:2003] Informação e documentação - Sumário -
%  Apresentação
%  \item[ABNT NBR 6028:2003] Informação e documentação - Resumo -
%  Apresentação
%  \item[ABNT NBR 6034:2004] Informação e documentação - Índice -
%  Apresentação
%  \item[ABNT NBR 10520:2002] Informação e documentação - Citações
%  \item[ABNT NBR 10719-2011] Informação e documentação - Relatório técnico
%  e/ou científico - Apresentação
%  \item[ABNT NBR 14724:2011] Informação e documentação - Trabalhos
%  acadêmicos - Apresentação
%  \item[ABNT NBR 15287:2011] Informação e documentação - Projeto de pesquisa -
%  Apresentação
%\end{description}

A classe \texttt{abntex2} foi criada como um conjunto de configurações da classe
\texttt{memoir}, cujas opções podem ser utilizadas. (Vide o manual da classe \texttt{memoir}.)


As opções mais comuns de inicialização do texto de um trabalho segundo as normas da ABNT são:

\begin{verbatim}
   \documentclass[12pt,openright,twoside,a4paper]{abntex2}
\end{verbatim} 

%A opção \verb+article+ é útil para produção de artigos com abn\TeX 2.
%Nesse caso, a maioria dos elementos pré-textuais se tornam desnecessários. A macro \verb+\part+ também é permitida com a opção\texttt{article}. Quando esta opção for
%utilizada, a classe \texttt{abntex2} não forçará quebra de página para os
%elementos pré-textuais e definirá a formatação do capítulo de forma idêntica à
%formatação das seções. Por padrão, quando a opção \texttt{article} estiver presente,
%você deve iniciar as divisões do trabalho com \verb+\section+, e não \verb+\chapter+, como
%é usual em trabalhos monográficos. Porém, caso queira iniciar as divisões com
%\verb+\chapter+ ao invés de \verb+\section+, adicione as linhas abaixo no preâmbulo do
%documento para que a numeração dos capítulos, seções, figuras e tabelas sejam
%corretamente sequenciados:
%
%\begin{verbatim}
%  \counterwithout{section}{section}
%  \counterwithout{figure}{chapter}
%  \counterwithout{table}{chapter}
%\end{verbatim}

O abn\TeX{} está programado para utilizar seja o compilador \texttt{pdflatex}, fazendo uso dos pacotes \texttt{babel} e \texttt{inputenc} (utf8) e fontes nativas do \LaTeX, ou então o compilador \texttt{xelatex}, fazendo uso do pacote \texttt{polyglossia} e \texttt{fontspec}, para uso de fontes TrueType e OpenType (ver capítulos \ref{fontes} e \ref{idiomas}).

\section{Macros de dados do trabalho}\label{sec-macrosdados}
O uso das macros do abn\TeX{} é bastante intuitivo, seguindo a filosofia do \LaTeX{} de não deixar o autor se preocupar com a formatação. As macros são bastante autoexplicativas:

\begin{verbatim}
\titulo{}

\autor{}

\local{}

\data{}

\orientador{}

\coorientador{}

\instituicao{}

\tipotrabalho{}

\preambulo{}

\imprimircapa

\imprimirfolhaderosto

...

\apendices

\anexos 

\end{verbatim}



Abaixo um exemplo do Modelo de Trabalho Acadêmico do abn\TeX:


\begin{verbatim}
% ---
% Informações de dados para CAPA e FOLHA DE ROSTO
% ---
\titulo{Modelo Canônico de\\ Trabalho Acadêmico com \abnTeX}
\autor{Equipe \abnTeX}
\local{Brasil}
\data{2013, v-1.3}
\orientador{Lauro César Araujo}
\coorientador{Equipe \abnTeX}
\instituicao{%
  Universidade do Brasil -- UBr
  \par
  Faculdade de Arquitetura da Informação
  \par
  Programa de Pós-Graduação}
\tipotrabalho{Tese (Doutorado)}
% O preambulo deve conter o tipo do trabalho, o objetivo, 
% o nome da instituição e a área de concentração 
\preambulo{Modelo canônico de trabalho monográfico acadêmico em conformidade com
as normas ABNT apresentado à comunidade de usuários \LaTeX.}


% ----------------------------------------------------------
% ELEMENTOS PRÉ-TEXTUAIS
% ----------------------------------------------------------
% \pretextual

% ---
% Capa
% ---
\imprimircapa
% ---

% ---
% Folha de rosto
% (o * indica que haverá a ficha bibliográfica)
% ---
\imprimirfolhaderosto*

\end{verbatim}

Ambientes abnTeX (a ser colocados entre \verb+\begin{*nome do ambiente*}+
\verb+\end{*nome do ambiente*}+):


\begin{multicols}{3}
\begin{itemize}
\item errata \item folhadeaprovacao \item dedicatoria \item agradecimentos \item epigrafe
\item resumo \item resumo[Abstract] \item listoffigures \item listoftables \item tableofcontents
\end{itemize}
\end{multicols}

Para mais informações e detalhes, referir-se à documentação do projeto abn\TeX 2: \url{http://code.google.com/p/abntex2/downloads/list}.