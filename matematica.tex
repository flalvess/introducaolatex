
Um dos motivos que levou D. Knuth a desenvolver o \TeX{} foi poder dar ao usuário, e não somente ao tipógrafo profissional, a capacidade de construir fórmulas e matemáticas com aparência profissional. Enquanto que  o \LaTeX, por si só, dá conta fórmulas matemáticas comuns, par fórmulas mais complicadas e outros recursos podemos utilizar um dos dois pacotes: \textsf{amsmath} ou \textsf{mathtools}. (O pacote \textsf{mathtools} estende as funcionalidades do \textsf{amsmath}.)

Elementos matemáticos podem ser inseridos de duas formas: no texto (\textit{inline}) ou destacado do texto (\textit{display}). Para inserir elementos matemáticos (equações etc.) juntamente com o texto, o modo mais fácil para alterar para o ``modo matemático'' é utilizar cifrões: \verb+$ a^2 - \sqrt{2} $+ = $ a^2 - \sqrt{2} $. Para elementos destacados do texto, pode-se utilizar o ambiente \texttt{displaymath} (\verb+\begin{displaymath}...\end{displaymath}+) ou o ``atalho'' -- para \LaTeX{}, \verb+\[...\]+, para \TeX{} \verb+$$...$$+. Utilizar este último é desencorajado, pois podem resultar problemas, particularmente com as macros do AMS-\LaTeX. Para inserir uma equação numerada, pode-se empregar o ambiente \texttt{equation}:

\begin{verbatim}
\begin{equation}
x = \dfrac{-b \pm \sqrt{b^2 - 4ac}}{2a}
\end{equation}
\end{verbatim}


\begin{equation}
x = \dfrac{-b \pm \sqrt{b^2 - 4ac}}{2a}
\end{equation}


Em modo matemático, insere-se sobrescritos com o acento circunflexo (\verb+^+) e subscritos com ``underline'' (\verb+_+)

Bons editores fornecem uma enorme quantidade de símbolos, e a maioria dos símbolos mais comuns tem um nome em inglês facilmente memorizável. Para a maior lista possível, vide \textit{The Comprehensive \LaTeX{} symbols list}, de Scott Pakin (\verb+texdoc symbols+ no prompt de comando ou, \textit{inter alia}, \url{http://www.las.ic.unicamp.br/pub/ctan/info/symbols/comprehensive/symbols-a4.pdf}). 

Por exemplo, \verb+\sqrt+ = \textit{square root}; \verb+\binom+ = binomial, \verb+\frac{num}{den}+ = \textit{fração, numerador e denominador}; \verb+\dfrac+ = \textit{display fraction}. Os comandos do \textsf{amsmath} \verb+\over+ (``sobre'') e \verb+\choose+ (``escolher'') equivalem, respectivamente, a frações e binômios:



\begin{verbatim}
{n! \over k!(n-k)!} = {n \choose k}
\end{verbatim}

\begin{displaymath}
{n! \over k!(n-k)!} = {n \choose k}
\end{displaymath}

\begin{verbatim}
\sqrt{\frac{a}{b}}
\end{verbatim}

\begin{displaymath}
\sqrt{\frac{a}{b}}
\end{displaymath}

\begin{verbatim}
\sqrt[n]{1+x+x^2+x^3+\ldots}
\end{verbatim}

\[ \sqrt[n]{1+x+x^2+x^3+\ldots} \]

Equações contínuas podem ser inseridas com o comando \verb+\cfrac+:

\begin{verbatim}

\[ 
  x = a_0 + \cfrac{1}{a_1
          + \cfrac{1}{a_2
          + \cfrac{1}{a_3 + \cfrac{1}{a_4} } } }
 \]

\end{verbatim}

\[ 
  x = a_0 + \cfrac{1}{a_1
          + \cfrac{1}{a_2
          + \cfrac{1}{a_3 + \cfrac{1}{a_4} } } }
 \]

\bigskip


Somas e integrais podem ser inseridas com \verb+\sum+ e \verb+\int+:

\begin{tabular}{cc}
\begin{minipage}{0.4\textwidth}
\verb+\sum_{i=1}^{10} t_i+
\end{minipage} & \begin{minipage}{0.4\textwidth}\[ \sum_{i=1}^{10} t_i \] \end{minipage}
\end{tabular}

\bigskip

É possível escrever partes que não estejam em itálico com o comando \verb+\mathrm{}+:



\begin{tabular}{cc}
\begin{minipage}{0.4\textwidth}
\verb+\int_0^\infty \mathrm{e}^{-x}\,\mathrm{d}x+
\end{minipage} & \begin{minipage}{0.4\textwidth}\[ \int_0^\infty \mathrm{e}^{-x}\,\mathrm{d}x \] \end{minipage}
\end{tabular}

\bigskip

Para especificar limites acima e abaixo de uma integral:



\begin{tabular}{cc}
\begin{minipage}{0.4\textwidth}
\verb+\int\limits_a^b+
\end{minipage} & \begin{minipage}{0.4\textwidth}\[ \int\limits_a^b \] \end{minipage}
\end{tabular}


Delimitadores -- parênteses, colchetes, chaves etc.:


\begin{tabular}{cc}
\begin{minipage}{0.4\textwidth}
\begin{verbatim}
( a ), [ b ], \{ c \}, | d |, \| e \|,
\langle f \rangle, \lfloor g \rfloor,
\lceil h \rceil, \ulcorner i \urcorner
\end{verbatim}
\end{minipage} & 
\begin{minipage}{0.4\textwidth}
\begin{displaymath}
( a ), [ b ], \{ c \}, | d |, \| e \|,
\langle f \rangle, \lfloor g \rfloor,
\lceil h \rceil, \ulcorner i \urcorner
\end{displaymath}
 \end{minipage}
\end{tabular}

\bigskip


Uma das praticidades do \LaTeX{} é que ele formata automaticamente o tamanho dos elementos e símbolos matemáticos, seja no texto ou destacado, de forma que o usuário não precisa de se preocupar com o tamanho do parêntese, da raiz etc. Para fórmulas com elementos de vários tamanhos, pode ser útil utilizar os comandos \verb+\left+ e \verb+\right+ para formatar o tamanho dos delimitadores (ver acima). 


\begin{tabular}{cc}
\begin{minipage}{0.4\textwidth}
\begin{verbatim}
\left(\frac{x^2}{y^3}\right)
\end{verbatim}
\end{minipage}
&
\begin{minipage}{0.4\textwidth}
\begin{displaymath}
\left(\frac{x^2}{y^3}\right)
\end{displaymath}
\end{minipage}
\end{tabular}

\bigskip

Matrizes podem ser inseridas, à maneira de tabelas (utilizando \& e \verb+\\+), com o ambiente\ldots{} \texttt{matrix}:

\bigskip

\begin{tabular}{cc}
\begin{minipage}{0.4\textwidth}
\begin{verbatim}
\begin{matrix}
  a & b & c \\
  d & e & f \\
  g & h & i
 \end{matrix}
\end{verbatim}
\end{minipage}
&
 \begin{minipage}{0.4\textwidth}
 \begin{matrix}
   a & b & c \\
   d & e & f \\
   g & h & i
  \end{matrix}
  \end{minipage}
\end{tabular}

\bigskip

Para especificar o alinhamento da matriz, utilize a versão com * (pacote \textsf{mathtools}):

\begin{verbatim}

\begin{matrix}
  -1 & 3 \\
  2 & -4
 \end{matrix}
 =
 \begin{matrix*}[r]
  -1 & 3 \\
  2 & -4
 \end{matrix*}
\end{verbatim}


\begin{matrix}
  -1 & 3 \\
  2 & -4
 \end{matrix}
 =
 \begin{matrix*}[r]
  -1 & 3 \\
  2 & -4
 \end{matrix*}
 

 O ambiente \texttt{cases} (pacote \textsf{amsmath}) permite escrever funções parte a parte:
 
 \begin{tabular}{cc}
\begin{minipage}{0.4\textwidth}
\begin{verbatim}

\[
 u(x) =
  \begin{cases}
   \exp{x} & \text{if } x \geq 0 \\
   1       & \text{if } x < 0
  \end{cases}
\]
\end{verbatim}
\end{minipage}
&
\begin{minipage}{0.4\textwidth}
\begin{minipage}{0.4\textwidth}
\end{minipage}
 \end{tabular}
 
 Ou então listas subdivididas com chaves (notar \verb+\text{}+ para texto dentro do modo matemático):
 
  \begin{tabular}{cc}
 \begin{minipage}{0.4\textwidth}
\begin{verbatim}
 $$
 \text{Maomé}
 \begin{dcases}
 \text{Abu Bakr}\\
 \text{Umar}\\
 \text{Uthman}\\
 \text{Ali}
    \begin{dcases}
    \text{Hassan}\\
    \text{Hussein}\\
    \end{dcases}
 \end{dcases}
 $$
\end{verbatim}
\end{minipage}
&

\begin{minipage}{0.4\textwidth}
 $$
 \text{Maomé}
 \begin{dcases}
 \text{Abu Bakr}\\
 \text{Umar}\\
 \text{Uthman}\\
 \text{Ali}
    \begin{dcases}
    \text{Hassan}\\
    \text{Hussein}\\
    \end{dcases}
 \end{dcases}
 $$
\end{minipage}
 \end{tabular}
 
 